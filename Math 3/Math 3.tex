\documentclass[twocolumn]{article}
\usepackage{amsmath}
\usepackage{amsfonts}
\usepackage{graphicx}

\begin{document}

\title{Math 3}
\author{Lecture notes by Rangi}
\date{\today}
\maketitle

\begin{abstract}
	Abstract Here
\end{abstract}

\section{Introduction}
	\paragraph{Previously:}
	we had equations like 
	\begin{equation}
	y ^{2}+4y+1=0 
	\end{equation}
	where the solution is a number.
	\begin{equation}
		\int_{a}^{b} f(x) dt = \text{number} = F(b)-F(a)	
	\end{equation}
	\paragraph{New:}
	$ f'(x) $ is given, determine $ f(x) $. The solution is a function.
	\begin{equation}
		\begin{split}
			\text{Velocity}(t) & = \text{Position}'(t)~~~~  \text{given}  \\
					   & = \text{Position}(t)~~~~~  \text{wanted}
		\end{split}
	\end{equation}
	\subsection{Differential equations}
	\paragraph{Example:}
	Interest rate
	\begin{equation}
	\begin{split}
		y(t):~~~~ & \text{assets at time }t \\
		\lambda<0:~~~~ & \text{constant interest rate}
	\end{split}
	\end{equation}
	[...]
	
	
\section{First-order DE}
\paragraph{For now:}
$ m=1,n=1 $ 
\paragraph{Consider:}
$ y'(x)=f(x,y(x)) $ explicit form of ODE\\\\
Function $ f $ is defined on
\begin{equation}
D=Dx\times Dy\in\mathbb R ^{2}
\end{equation}
\paragraph{Ex.}
Strip: [...]
\subsection{Geometric interpretation: Direction field}
[...]
\subsection{Observations}
\begin{enumerate}
	\item Through each point $ (x_0,y_0)\in D $ there passes exactly one solution
	\item Each solution curve is maximal (no blow up)
	\item Solution curves don't intersect 
\end{enumerate}
\subsection{Existence and Uniqueness of a solution to ODE}
	JVP:$~~~~ y'=f(x,y),y(x_0)=y_0,$ domain D
	\paragraph{Theorem (Peano):}
	Assume that $f$ is continuous on $D$ and $(x_0,y_0)\in D$. Then if JVP has at least one solution. This solution is maximal ... we can continue solv. until the boundary D.
	\paragraph{Theorem (PicardLindelöt):}
	Let $f$ be continuous on $D$ and let $f$ be cont. diff. with respect $y$. ($\rightarrow$ Lipschitz cont.) Let $(x_0,y_0)\in D$. Then the IVP has a unique solution.
	\paragraph{Ex.:}
	\begin{equation}
		\begin{split}
		(x ^{2}-x)y'=(2x-1)y ~~~~~~&\text{implicit form}\\
		y'=\frac{2x-1}{x ^{2}-x}*y ~~~~~& \text{explicit form}
	\end{split}
	\end{equation}
	[...]
	\begin{enumerate}
		\item $x_0\not\in \{0;1\}\rightarrow$ unique sol. (due to Picard-Lindelöf Thm.)
		\item $x_0\in \{0;1\}\rightarrow$ infinitely many solutions
	\end{enumerate}	

	
	

\end{document}
